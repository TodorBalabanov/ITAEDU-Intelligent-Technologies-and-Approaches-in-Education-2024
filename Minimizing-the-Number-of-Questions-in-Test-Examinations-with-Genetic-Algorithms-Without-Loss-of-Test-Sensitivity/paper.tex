\documentclass[aip,cp,amsmath,amssymb,reprint,]{revtex4-2}

\usepackage{graphicx}
\usepackage{dcolumn}
\usepackage{bm}

\usepackage[utf8]{inputenc}
\usepackage[T1]{fontenc}

\usepackage{mathptmx} 

\begin{document}

\title{Minimizing the Number of Questions in Test Examinations with Genetic Algorithms Without Loss of Test Sensitivity}

\author{Gergana Mateeva}
 \email{gergana.mateeva@iict.bas.bg}
\author{Delyan Keremedchiev}
 \email{delyan.keremedchiev@iict.bas.bg}
\author{Petar Tomov}
 \email{petar.tomov@iict.bas.bg}
\author{Iliyan Zankinski}
 \email{iliyan.zankinski@iict.bas.bg}
\author{Plamen Petrov}
 \email{plamen.petrov@iict.bas.bg}
\author{Todor Balabanov}
 \email[Corresponding author: ]{todor.balabanov@iict.bas.bg}
\affiliation{Bulgarian Academy of Sciences\\ Institute of Information and Communication Technologies\\ acad. Georgi Bonchev Str. block 2\\ 1113 Sofia\\ Bulgaria}

%\date{\today}

\begin{abstract}
The reasonable number of questions for an exam with closed questions depends on various factors, such as the complexity of the material being assessed, the time allotted for the exam, and the desired level of detail in evaluating students' understanding. Test sensitivity is strongly correlated with the number of questions in the questionnaire. With all other criteria equal, the number of questions is crucial for test measurement quality. For both sides, minimizing the number of questions is cost-efficient but without significant loss of sensitivity. It is a discrete optimization problem where genetic algorithms can appear very useful.
\end{abstract}

\maketitle

\section{Introduction}

Testing with a predefined set of possible answers is popular and widely accepted. Due to this popularity, optimizing the testing process becomes highly relevant. The examiner must be confident in the thorough examination of the examinees while simultaneously ensuring that the examinees are not subjected to undue pressure. Both requirements can be met by employing shorter questionnaires sensitive to the examinees' skills and knowledge.

Consideration must be given to time constraints, including the available time for the exam. It's crucial to strike a balance, ensuring that examinees are provided with ample time to thoroughly read and respond to each question within the allocated timeframe. Equally important is the content coverage. The exam should adequately address the material taught in the course or lesson. The number of questions should be sufficient to evaluate students' comprehension of key concepts and topics. As a rough guideline, exams with closed questions are commonly comprised of around 20 to 50 questions, although this can vary depending on different factors.

The sensitivity of a closed-answer test is determined by its ability to measure differences in knowledge or understanding among test-takers accurately. The number of questions significantly influences the impact on the test's sensitivity in the questionnaire. Increased precision is attained as the number of questions in a test is raised, improving sensitivity. This improvement is attributed to the broader assessment of the test-taker's knowledge or abilities facilitated by more questions. A more comprehensive evaluation of individual differences is enabled as the test covers a broader range of topics and assesses various aspects of the subject matter. Sampling errors are reduced with more questions, as these errors, driven by random variability rather than actual differences in knowledge or ability among test-takers, decrease. The impact of random guessing or chance fluctuations in performance is minimized with more questions, resulting in more reliable and precise results. Enhanced discrimination between high-performing and low-performing individuals is achieved through more questions, each offering an opportunity to differentiate between test-takers with varying knowledge or understanding levels. This increased discrimination contributes to more accurate rankings or classifications of individuals based on their test performance. The susceptibility to guessing in closed-answer tests, such as multiple-choice exams, is related to their robustness. However, as the number of questions increases, the influence of guessing on overall test scores diminishes. With a larger pool of questions, the likelihood of guessing correctly on a significant portion of the test decreases, rendering the test results more reflective of actual knowledge or understanding. Overall, the sensitivity of a closed-answer test is enhanced by the number of questions, enabling a more thorough and precise assessment of test-takers knowledge or abilities, reducing the influence of random variability, enhancing discrimination between individuals, and mitigating the impact of guessing. As a disadvantage, increasing the number of questions means longer exams and higher resource usage.

\section{Conclusion}

\begin{acknowledgments}
This research is conducted as part of project No. KP-06-M75/3 ''Research of methods and technologies for digitalization of education'', financed by the Bulgarian National Science Foundation (NSF).
\end{acknowledgments}

\nocite{*}
\bibliography{aipsamp}

\end{document}
